\section{Conclusions personnelles}
\subsection{Herrier Lucie}
A la fin de ce projet, j'ai l'impression de mieux maîtriser l'outil Matlab. J'ai découvert d'autres fonctionnalités que celles vues en classe.  C'était un projet intéressant et agréable à réaliser, car la musique fait partie de ma vie depuis que je suis toute petite. J'ai parfois eu un peu du mal avec le code, notamment lors de la réalisation de l'interface graphique.
 Par ailleurs, je suis contente que notre groupe ait réussi à réaliser un accordeur de guitare suffisamment efficace. Je pense que nous pouvons tous être fier de ce que nous avons produit. Toutefois, une amélioration de l'accordeur le rendrait encore plus efficace.


\subsection{Juckler Christian}
Ce projet m'a intéressé sur plusieurs aspects. 
D'abord sur un aspect technique, l'analyse spectrale du son m'a toujours intéressée, mais je n'avais jamais pu y consacrer du temps. 
C'est chose faite avec ce projet.
Ensuite sur un aspect musical, mon groupe m'a appris quelques bases en musique.
Ces bases m'ont permis de mieux comprendre le principe de l'accordeur, vu que je n'avais aucune connaissance en musique.
Ce projet m'a permis aussi de me rendre compte qu'il n'est pas facile de travailler dans un domaine inconnu. 
Finalement, je suis satisfait du travail accompli. Notre accordeur fonctionne et peut facilement être amélioré.

\subsection{Musuvaho Grace}
J'aimais bien le sujet du projet et j'étais super motivée, mais j'ai eu du mal à apporter une réelle contribution au travail. J'avoue que je me suis sentie parfois un peu inutile. J'ai aidé à faire des recherches et essayer de trouver les bugs. Malgré tout, grâce aux recherches, j'ai pu en apprendre d'avantage sur le traitement du son avec Matlab.

\subsection{Nyssens Sylvain}
Nécessitant de maîtriser différentes notions vue aux cours, ce projet m'a permis de mieux assimiler la théorie vue en cours. Tout cela en illustrant un cas pratique qui m'intéresse particulièrement, étant guitariste. La prise en main de GUIDE, l'outil de création d'interface graphique de MatLab, n'a pas été simple aux premiers abords. Mais celui-ci s'est finalement révélé assez flexible et à permis de créer l'interface désirée sans trop d'encombres. Les interactions au seins du groupe ont été enrichissante et ont, selon moi, donné un beau résultat.