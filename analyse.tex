\section{Méthodologie utilisée}
Lors de la réalisation de notre accordeur, nous sommes passés par plusieurs étapes avant d'arriver au produit actuel. Cette partie du rapport explique ces différentes phases de développement, ainsi que les difficultés rencontrées au cours de celles-ci.\\

En premier lieu, nous avons tout d'abord cherché à enregistrer un son dans Matlab. En effet, nous voulions pouvoir visualiser le spectre fréquentiel de l'enregistrement du son produit par une corde de guitare sur une durée déterminée.
Après plusieurs tentatives, nous y sommes arrivé.
Nous nous sommes vite rendu compte qu'il y avait un problème dans notre analyse.
Nous avions sous-estimé la puissance des harmoniques.
Lors de nos test, nous travaillions avec un logiciel capable de simuler un diapason et une guitare.
Avec le diapason, nous n'avions aucun doute sur notre approche. En effet, les fréquences émises étant unique, nous avions un spectre fréquentiel net avec la fréquence jouée se démarquant nettement.
Toutefois, avec la guitare, nous avons remarqué que les harmoniques étaient plus puissante que la fréquence fondamentale. Nous nous retrouvions donc face à un problème.\\

Dans un deuxième temps, nous donc avons cherché à éliminer les harmoniques. En effet, les harmoniques sont des fréquences indésirables et parasites empêchant un accordage efficace.
Pour ce faire, nous avons analysé les différents accords existants ainsi que la fréquence maximale à détecter pour la corde la plus aiguë.
Il en a résulté que nous pouvions nous passer des fréquences au-delà de \numprint[Hz]{500}.
Nous avons ainsi cherché à limiter la représentation graphique autour de la fréquence voulue.
Chaque note possédant sa propre fréquence, il n'est pas compliqué de borner les fréquences représentées sur le graphique autour de celle recherchée. Nous nous sommes rendus compte que pour éviter toute harmonique indésirable sur la représentation spectrale, il suffisait d'interdire les fréquences supérieures à celle recherchée, plus 50Hz. La fréquence enregistrée et voulue devenait donc la plus puissante, donc plus facile à récupérer. Évidemment, cela a également apporté un problème, de sorte que si l'on joue un La à 220Hz alors qu'on veut accorder un Mi à 83Hz, la fréquence jouée n'est plus détectable. Nous avons alors décidé que ce n'était pas de la plus grande importance, une corde de guitare étant rarement 50Hz plus haut ou plus bas lorsqu'elle doit être accordée. \\

Par la suite, nous avons réfléchi à une façon de stocker les différents accords existants en associant les notes avec leur fréquence respective.
Pour ce faire, nous sommes partis d'un La \numprint[Hz]{440} afin de générer la gamme de la 3ème octave. À partir de cette gamme, nous avons pu calculer les fréquences des notes pour les différents accords. Nous avons tout d'abord généré un accordage classique en E, puis nous sommes partis vers d'autres accord également fréquemment rencontrés, tels que le Drop D ou encore l'Open D.

Parallèlement, nous avons commencé une interface graphique, afin de rendre l'accordeur intuitif pour l'utilisateur.
Cette première interface affichait une zone d'accordage rectangulaire avec un graphique s'actualisant à chaque enregistrement, pour une fréquence fixée.
Avec cette interface, nous pouvions analyser une note précise durant un laps de temps. À la fin de ce laps de temps, nous affichions la fréquence réelle de la note jouée. Selon l'écart avec la valeur théorique, nous pouvions visualiser si l'accordage était correct, trop grave ou trop aigu. Après avoir limité l'enregistrement à un laps de temps, nous sommes passés au temps réel. Nous lancions l'enregistrement, et à l'aide d'une boucle while, nous analysions continuellement le son enregistré. \\

Finalement, nous avons crée une interface graphique plus évoluée à l'aide de l'outil GUIDE. Nous avons ensuite assemblé l'interface et la fonction de génération des fréquences des accords et d'enregistrement pour arriver à un programme fonctionnel. Nous avons dû changer les paramètres de la boucle while avec cette nouvelle interface. Au départ, nous pensions lancer l'enregistrement, et la boucle, au clic d'un bouton, et le terminer au clic sur une autre note, sauf que nous n'y sommes pas arrivés. Nous avons donc décidé de définir un timer de 5 secondes lors d'un clic sur une note, pendant lequel l'accordage serait possible. 
L'interface proposée aujourd'hui pour notre accordeur permet de changer d'accord, de jouer les notes, et d'accorder, bien évidemment.


% A compléter