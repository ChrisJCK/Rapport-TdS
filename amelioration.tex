\section{Améliorations possibles}
Notre projet d'accordeur pourrait être amélioré de plusieurs manières. Tout d'abord, nous pourrions rajouter plusieurs autres accords à la liste de ceux déjà proposés. Cela permettrait à l'accordeur d'être encore plus efficace pour les guitaristes aimant jouer plusieurs styles musicaux différents.\\

Nous avions également pensé, à la base, à laisser l'accordage de la corde actif jusqu'à ce qu'une autre note soit sélectionnée. N'ayant pas réussi à mettre cette technique en place, nous avons opté pour l'implémentation d'un timer. Un amélioration possible serait donc de revenir à notre idée initiale d'accordage en continu pour une corde. Un clic sur la note lancerait le processus d'accordage, et un clic sur une autre note stopperait le premier et lancerait le nouveau.\\

Enfin, nous aurions pu réaliser une interface permettant, une fois l'accord sélectionné, d'accorder n'importe quelle corde. C'est-à-dire qu'au lieu des notes sur la droite et d'un graphique à gauche, nous aurions eu 6 graphiques (un par note). Nous n'avons pas pu mettre en place un tel système à cause du problème des harmoniques mentionné plus haut. Cependant, une amélioration de l'accordeur pour parvenir à une telle version serait intéressante. En effet, plus besoin de choisir la note. Le guitariste jouerait la corde qu'il veut, et il lui suffirait de regarder le graphique correspondant pour l'accorder.